\documentclass{article}

%%\setlength{\oddsidemargin}{0.25in}
%%\setlength{\textwidth}{6in}
%%\setlength{\topmargin}{0.5in}
%%\setlength{\textheight}{9in}

%\renewcommand{\baselinestretch}{1.5} 

\usepackage{amsmath,amsfonts,amsthm,amssymb,amscd}
\usepackage[margin=1.15in]{geometry}
\usepackage{mathrsfs}
\usepackage{bbm}
\usepackage{bm}
\usepackage{mathtools}
\usepackage{cancel}
\usepackage{listings}
\usepackage{booktabs}

% Begin/End
\newcommand{\be}{\begin{equation}}
\newcommand{\ee}{\end{equation}}
\newcommand{\benn}{\begin{equation*}}
\newcommand{\eenn}{\end{equation*}}

% Theorem-like Environments
\newtheorem{thm}{Theorem}[section]
\newtheorem{conj}[thm]{Conjecture}
\newtheorem{cor}[thm]{Corollary}
\newtheorem{lem}[thm]{Lemma}
\newtheorem{prop}[thm]{Proposition}
\newtheorem{exa}[thm]{Example}
\newtheorem{defi}[thm]{Definition}
\newtheorem{exe}[thm]{Exercise}
\newtheorem{rek}[thm]{Remark}
\newtheorem{que}[thm]{Question}
\newtheorem{prob}[thm]{Problem}
\newtheorem{cla}[thm]{Claim}
\newtheorem{esti}[thm]{Estimation}
\newtheorem{assu}[thm]{Assumption}

% Two-Case and Three-Case Definitions
\newcommand{\twocase}[5]{#1 \begin{cases} #2 & \text{#3}\\ #4
&\text{#5} \end{cases}   }
\newcommand{\threecase}[7]{#1 \begin{cases} #2 &
\text{#3}\\ #4 &\text{#5}\\ #6 &\text{#7} \end{cases}   }

% Fractions
\newcommand{\foh}{\frac{1}{2}}
\newcommand{\foq}{\frac{1}{4}}
\newcommand{\fon}{\frac{1}{n}}

% Blackboard Letters
\newcommand{\R}{\ensuremath{\mathbb{R}}}
\newcommand{\C}{\ensuremath{\mathbb{C}}}
%\newcommand{\Z}{\ensuremath{\mathbb{Z}}}
\newcommand{\Q}{\mathbb{Q}}
\newcommand{\N}{\mathbb{N}}
\newcommand{\F}{\mathbb{F}}
%\newcommand{\W}{\mathbb{W}}
\newcommand{\sph}{\mathbb{S}}

% Greek Letters
\newcommand{\ga}{\alpha}
\newcommand{\gb}{\beta}
\newcommand{\gep}{\epsilon}
\newcommand{\vgep}{\varepsilon}
\newcommand{\gth}{\theta}
\newcommand{\vgth}{\vartheta}
\newcommand{\gk}{\kappa}
\newcommand{\gl}{\lambda}
\newcommand{\gs}{\sigma}

\newcommand{\bgb}{\bs{\gb}}

% Matrix
\newcommand{\mattwo}[4]
{\left(\begin{array}{cc}
                        #1  & #2   \\
                        #3 &  #4
                          \end{array}\right) }
\newcommand{\matthree}[9]
{\left(\begin{array}{ccc}
                        #1  & #2 & #3  \\
                        #4 &  #5 & #6 \\
                        #7 & #8 & #9 \\
                          \end{array}\right) }

% Linear Algebra
\newcommand{\col}{\mathrm{col} \,}
\newcommand{\im}{\mathrm{im} \,}
\newcommand{\rank}{\mathrm{rank} \,}
\newcommand{\trace}{\mathrm{tr} \,}
\newcommand{\laspan}{\mathrm{span} \,}
\newcommand{\iprod}[2]{\big\langle #1, \, #2 \big\rangle}
\newcommand{\tr}{\mathrm{T}}
\newcommand{\diag}{\mathrm{diag}}
\newcommand{\Id}{\bs{I}}

% Calculus
\newcommand{\dd}[2]{\frac{d#1}{d#2}}
\newcommand{\pdd}[2]{\frac{\partial #1}{\partial #2}}

% Operators
\newcommand{\Ad}{\mathrm{Ad}}
\newcommand{\Lie}[1]{\mathrm{Lie}(#1)}
\newcommand{\sign}{\mathrm{sign} \,}
\newcommand{\supp}{\mathrm{supp} \,}

% Geometry
\newcommand{\del}[2]{\nabla_{#1}#2}
\newcommand{\multi}[1]{\underset{\raisebox{2pt}[0pt]{$\rightharpoondown$}}{#1}}
\newcommand{\multis}[1]{\underset{\raisebox{2pt}[2pt]{$\scriptscriptstyle\rightharpoondown$}}{#1}}

% Probability/Statistics
\newcommand{\Prb}{\mathrm{P}}
\newcommand{\E}{\mathrm{E}}
\newcommand{\Ehat}{\hat{\E}}
\newcommand{\Ex}{\mathbb{E}}
\newcommand{\bias}{\mathrm{bias}}
\newcommand{\var}{\mathrm{var}}
\newcommand{\cov}{\mathrm{cov}}
\newcommand{\cid}{\overset{d}{\to}}
\newcommand{\X}{\bs{X}}
\newcommand{\Y}{\bs{Y}}
\newcommand{\SE}{\mathrm{SE}}

% Distributions
\newcommand{\Bern}{\mathrm{Bernoulli}}
\newcommand{\Poisson}{\mathrm{Poisson}}
\newcommand{\Normal}{\mathrm{N}}
\newcommand{\DGamma}{\mathrm{Gamma}}
\newcommand{\Unif}{\mathrm{Uniform}}

% Math operators
\DeclareMathOperator*{\argmin}{arg\,min}

% Formatting
\newcommand{\bs}[1]{\boldsymbol{#1}}

% Graphics
\newcommand{\grapher}[2]{\centerline{\includegraphics[width=#2]{#1}}}

%%%%%%%%%%%%%%%%%%%%%%%%%%%%%%%%%%%%%%%%%%%%%%%%%%
% Code Environments

\usepackage{color}

\definecolor{dkgreen}{rgb}{0,0.6,0}
\definecolor{gray}{rgb}{0.5,0.5,0.5}
\definecolor{mauve}{rgb}{0.58,0,0.82}

\usepackage[normalem]{ulem}
\usepackage{algpseudocode}
\usepackage{graphicx} 
\usepackage{fancyvrb} 
\lstnewenvironment{Rcode}{\lstset{frame=tb,% setup listings 
        language=R,% set programming language 
        basicstyle=\small\ttfamily,% basic font style 
        keywordstyle=\bfseries\color{blue},% keyword style 
        commentstyle=\ttfamily\itshape\color{dkgreen},% comment style 
          stringstyle=\color{mauve},
        numbers=left,% display line numbers on the left side 
        numberstyle=\scriptsize,% use small line numbers 
        numbersep=10pt,% space between line numbers and code 
        tabsize=3,% sizes of tabs 
        showstringspaces=false,% do not replace spaces in strings by a certain character 
        captionpos=b,% positioning of the caption below 
        breaklines=true,% automatic line breaking 
        escapeinside={(*}{*)},% escaping to LaTeX 
        fancyvrb=true,% verbatim code is typset by listings 
        extendedchars=false,% prohibit extended chars (chars of codes 128--255) 
        literate={"}{{\texttt{"}}}1{<-}{{$\leftarrow$}}1{<<-}{{$\twoheadleftarrow$}}1 
        {~}{{$\sim$}}1{<=}{{$\le$}}1{>=}{{$\ge$}}1{!=}{{$\neq$}}1{^}{{$^\wedge$}}1,% item to replace, text, length of chars 
        alsoletter={.<-},% becomes a letter 
        alsoother={$},% becomes other 
        otherkeywords={!=, ~, $, *, \&, \%/\%, \%*\%, \%\%, <-, <<-, /},% other keywords 
        deletekeywords={c}% remove keywords }}{}
        }
        }{}

\numberwithin{equation}{section}
\begin{document}

%%%%%%%%%%%%%%%%%%%%%%%%%%%%%%%%%%%%%%%%%%%%%%%%%
% MACROS

\newcommand{\tp}{\mathrm{T}}

\newcommand{\z}{\bs{z}}
\newcommand{\x}{\bs{x}}
\newcommand{\w}{\bs{w}}
\newcommand{\inst}{\bs{u}}
\newcommand{\instc}{u}

\newcommand{\pz}{{p_{\z}}}
\newcommand{\pw}{{p_{\w}}}
\newcommand{\px}{{p_{\x}}}
\newcommand{\pinst}{{p_{\inst}}}

\newcommand{\reg}{\gth}
\newcommand{\eft}{\gth}
\newcommand{\efthat}{\hat{\eft}}
\newcommand{\nuis}{\bs{\gamma}}
\newcommand{\sdh}{\gs_h}
\newcommand{\sdv}{\gs_v}
\newcommand{\covhv}{\gs_{hv}}

\newcommand{\efc}{\bs{\gamma}}
\newcommand{\W}{\bs{W}}
\newcommand{\Z}{\bs{Z}}
\newcommand{\erry}{\gep}
\newcommand{\pd}{\partial}

\newcommand{\gbt}{\bs{\gb}}
\newcommand{\argb}{\bs{b}}

%\newcommand{\fix}[1]{\smash{\underset{{\bs{-}}}{#1}}}
%\newcommand{\fix}[1]{\underline{#1}}


\noindent David Gold \hfill \today \\
STAT 572 --- Questions for mentor (Adrian Dobra) \vspace{.2em}\hrule
\vspace{1em}

\noindent The following questions concern the motivation and examples that \cite{BCH11} give for their condition ASM on page 3-7. We recall their set-up. Suppose for $i\in1,\ldots,n$ that
\benn
	y_i \ = \ f(\z_i) + \gep_i,
\eenn
where $y_i \in \R, \z_i\in\R^\pz$, and $\gep_i \sim_{iid} \Normal(0,\gs^2)$. Let $\x_i = P(\z_i) = \big(P_1(\z_i),\ldots,P_\px(\z_i)\big) \in \R^\px$ be a vector-valued transformation of $\z_i$. Say that $f(\z)$ is \emph{approximately sparse} if there exists a $\gbt \in\R^{\px}$ so that $s = \|\gbt\|_0$ (that is, $s$ is the number of non-zero elements of $\gbt$) and 
\benn
	r_i \ := \ f(\z_i) - \langle \x_i, \gbt \rangle
\eenn
satisfy
\begin{enumerate}
\itemsep0em
	\item  $c_s := \left[n^{-1}\sum_i (r_i)^2\right]^{1/2} \leq K\gs\sqrt{s/n}$ for a universal constant $K$;
	\item  $s \equiv s(n) = o(n/\log\px)$. 
\end{enumerate}
(All sums taken over the index $i$ are evaluated over $i\in1,\ldots,n$ unless noted otherwise.)

\newcommand{\gd}{\delta}
\vspace{1em}
\noindent \textbf{1.}\quad The authors mention one can identify such a $\gbt$ as the solution to the optimization problem
\be\label{eq1}
	\bgb \ \in \ \argmin_{\argb\,\in\,\R^{\px}} \Big\{\underbrace{\fon\sum_i \big(\underbrace{f(\z_i) - \langle\x_i,\argb\rangle}_{r_i}\big)^2}_{c_s^2} + \gs^2\|\argb\|_0/n \Big\} \,.
\ee
They mention that, in ``common nonparametric problems'', the optimal solution $\gbt$ balances the order of ``approximation error'' $c_s^2$ and the order of the ``estimation error'' $\gs^2\|\argb\|_0/n$. Why is this the case? Is it actually a condition that the solution must satisfy this ``balancing'', or is this a restriction imposed by the authors? If the former, where can one find a reference for this phenomenon?

\vspace{1em}
\noindent\textbf{2.}\quad One example of a ``nonparametric problem'' that \cite{BCH11} study (page 6-7) is that of an $f$ for which a Fourier expansion $f(\z) = \sum_{j=1}^{\infty} \gd_j P_j(\z_i)$, where the $P_j(\z)$, $j\in\N$ are orthonormal basis functions on $\mathcal{Z} = [0,1]^{\pz}$ and the $\z_i$ are assumed distributed uniformly on $\mathcal{Z}$. If one assumes that the Fourier coefficients decay as $\gd_j \propto j^{-\nu}$, where $\nu$ is a smoothness measure of $f$, and one considers $\gbt = (\gd_j)_{j=1}^{d}$ and $\x(\z) = (P_1(\z), \ldots, P_d(\z))$, then $c_s$ as defined above satisfies
\benn
	c_s \ = \ O_P(\sqrt{\E[f(\z) - \langle \x, \gbt\rangle]})\,, \qquad\qquad \sqrt{\E[f(\z) - \langle \x, \gbt\rangle]} \ \lesssim \ d^{1/2-\nu} \,.
\eenn
(I have their $K$ to a $d$ since the authors use $K$ previously for the different, conflicting purpose identified above.) The authors state that ``Balancing the order $d^{1/2-\nu}$ of approximation error with the order $\sqrt{d/n}$ of the estimation error gives the oracle-rate-optimal number of series terms $s = d \propto n^{1/(2\nu)}$. My first question is: is this an instance where the authors are citing the fact that the optimal solution to \eqref{eq1} must balance the orders of the two terms, or are they imposing that restriction themselves and then deriving conditions on the relevant quantities from that self-imposed restriction? My second question is: is it even true that $\gbt$ in this instance satisfies \eqref{eq1}? If so, how does one show this? Finally, why do the authors consider random $\z_i$ in this example but explicitly consider the $\z_i$ (and hence the $\x_i$) fixed in their definition of condition ASM (essentially the definition of approximate sparsity above) on page 3? Is this purposeful, or sloppiness? 






\bibliographystyle{plain}
\bibliography{../references}
%\bibliography{pres_intro}



\end{document}